%%%%%%%%%% file participatory-gis_notizen.tex %%%%%%%%%
%   Partizipative GIS - Anwendung im globalen Süden   %
%%%%%%%%%%%%%%%%%%%%%%%%%%%%%%%%%%%%%%%%%%%%%%%%%%%%%%%


\documentclass[14pt,a5paper,landscape]{handout}

\usepackage[a5paper,landscape,top=2cm, bottom=2cm, left=1cm, right=1cm]{geometry}

\usepackage{amssymb}
\setcounter{tocdepth}{3}
\usepackage{graphicx}

\usepackage{hyperref}               % for hyperlink references
\usepackage[ngerman]{babel}
\usepackage[utf8]{inputenc}			    % for working umlaute
\usepackage[T1]{fontenc}            % wichtig für Trennung von Wörtern mit Umlauten
\usepackage{microtype}						  % verbesserter Randausgleich
\usepackage{pdfpages}

\usepackage[babel,german=quotes]{csquotes}

\addto\extrasngerman{\def\figureautorefname{Abb.}}		% changes figure reference text to "Abb."

\usepackage{float}							        							% for use of "H" specifier in floats
\usepackage[section]{placeins}          							% keep floats (images, tables, ..) in their place

% change symbols for unordered lists (itemize)
\renewcommand{\labelitemi}{$\bullet$}
\renewcommand{\labelitemii}{$\circ$}

\usepackage{titlesec}
\titleformat{\section}{\large\bfseries}{\thesection}{1em}{}
\titleformat{\subsection}{\large\bfseries}{\thesection}{0.8em}{}

\SetInstructor{Patrick Schulz \& Simon Hötten}
\SetCourseTitle{Geodatenbanken}
\SetHandoutTitle{Der R*-Baum}
\SetSemester{SS 2015}
\SetDate{02. Juli 2015}

\begin{document}

% Dokument
%%%%%%%%%%%%%%%%%%%%%%%%%%%%%%%%%%%%%%%%%%%%%%%%%%%%%%%%%%%%%%%%%%%%


\section{Optimierungskriterien} % (fold)
\label{sec:optimierungskriterien}

\begin{description}
	\item[Flächenausnutzung maximieren] Verbesserte Performance, weil Entscheidungen auf höheren Ebenen des Baumes getroffen werden können
	\item[Überlappung minimieren] Reduziert die Anzahl an zu durchsuchenden Teilbäumen
	\item[Summe der Kantenlänge minimieren] Quadratischere Rechtecke -> verbessertes Packen -> kleinere MBRs
	\item[Speichernutzung maximieren] Wahrscheinlichkeit für Splits verringern -> flacherer Baum -> schnelleres Traversieren
\end{description}

% section optimierungskriterien (end)


\end{document}
